\documentclass[a4paper,11pt]{paper}

\usepackage[utf8]{inputenc}
\usepackage[T1]{fontenc}
\usepackage[margin=3.2cm]{geometry}
\usepackage{enumitem}
\usepackage{CJKutf8}
\usepackage[colorlinks=true,urlcolor=blue,linkcolor=black]{hyperref}
\usepackage{mathtools}
\usepackage{listings}
\usepackage{fancyvrb}
\usepackage{enumitem}
\usepackage{tikz}
\usepackage{listings}
\usepackage{xcolor}
\usepackage{amsmath}
\usepackage{calc}
\usepackage{relsize}
\usepackage{emoji}  % lualatex
\usepackage{fontawesome}  % lualatex
\usepackage{fancyvrb}

\usepackage{lastpage}
\usepackage{fancyhdr}
\pagestyle{fancy}
\fancyhf{} % clear existing header/footer entries
% Place Page X of Y on the right-hand
% side of the footer
\fancyfoot[R]{Page \thepage \hspace{1pt} of \pageref{LastPage}}

\usetikzlibrary{calc,shapes.multipart,chains,arrows}

\renewcommand*{\theenumi}{\thesection.\arabic{enumi}}
\renewcommand*{\theenumii}{\theenumi.\arabic{enumii}}
\let\orighref\href
\renewcommand{\href}[2]{\orighref{#1}{#2\,\smaller[4]\faExternalLink}}

\let\Red=\alert
\definecolor{few-gray-bright}{HTML}{010202}
\definecolor{few-red-bright}{HTML}{EE2E2F}
\definecolor{few-green-bright}{HTML}{008C48}
\definecolor{few-blue-bright}{HTML}{185AA9}
\definecolor{few-orange-bright}{HTML}{F47D23}
\definecolor{few-purple-bright}{HTML}{662C91}
\definecolor{few-brown-bright}{HTML}{A21D21}
\definecolor{few-pink-bright}{HTML}{B43894}

\definecolor{few-gray}{HTML}{737373}
\definecolor{few-red}{HTML}{F15A60}
\definecolor{few-green}{HTML}{7AC36A}
\definecolor{few-blue}{HTML}{5A9BD4}
\definecolor{few-orange}{HTML}{FAA75B}
\definecolor{few-purple}{HTML}{9E67AB}
\definecolor{few-brown}{HTML}{CE7058}
\definecolor{few-pink}{HTML}{D77FB4}

\definecolor{few-gray-light}{HTML}{CCCCCC}
\definecolor{few-red-light}{HTML}{F2AFAD}
\definecolor{few-green-light}{HTML}{D9E4AA}
\definecolor{few-blue-light}{HTML}{B8D2EC}
\definecolor{few-orange-light}{HTML}{F3D1B0}
\definecolor{few-purple-light}{HTML}{D5B2D4}
\definecolor{few-brown-light}{HTML}{DDB9A9}
\definecolor{few-pink-light}{HTML}{EBC0DA}

\colorlet{alert-color}{few-red-bright!80!black}
\colorlet{comment}{few-blue-bright}
\colorlet{string}{few-green-bright}

\lstdefinestyle{ccode}{
    showstringspaces=false,
    stringstyle={\ttfamily\color{string}},
    language=C,escapeinside=`',columns=flexible,commentstyle=\color{comment},
    basicstyle=\ttfamily,
    classoffset=2, keywordstyle=\color{alert-color}
}

\lstnewenvironment{ccode}[1][]%
    {\lstset{style=ccode,basicstyle=\ttfamily\openup-.17\baselineskip,#1}}%
    {}

\lstset{
  basicstyle=\itshape,
  xleftmargin=3em,
  literate={->}{$\rightarrow$}{2}
           {α}{$\alpha$}{1}
           {δ}{$\delta$}{1}
           {ε}{$\epsilon$}{1}
}

\renewcommand{\baselinestretch}{1.1}
\setlength{\parindent}{0pt}
\setlength{\parskip}{1em}

\title{INF333 2024-2025 Spring Semester}
\author{
\textbf{\color{teal}{Pintocchio}} 
\\ Sude Melis Pilaz <22401992@ogr.gsu.edu.tr>
\\ Ali Burak Saraç <21401932@ogr.gsu.edu.tr>}

\begin{document}

\maketitle

\section*{\LARGE Homework III \\
Design Document}

Please provide answers inline in a \texttt{quote} environment.


\section{Preliminaries}

\textbf{Q1:} If you have any preliminary comments on your submission, notes for the TAs, or extra credit, please give them here.
\begin{quote}
  Answer here
\end{quote}


\textbf{Q2:} Please cite any offline or online sources you consulted while preparing your
submission, other than the Pintos documentation, course text, and lecture notes.
\begin{quote}
  Answer here
\end{quote}


\newpage
\section{Page Table Management}
\subsection{Data Structures}
\textbf{Q3:} Copy here the declaration of each new or changed `struct' or `struct' member, global or static variable, `typedef', or enumeration. Identify the purpose of each in 25 words or less.
\begin{quote}
\textbf{From page.h:}
\begin{verbatim}
enum vm_type { VM_BIN, VM_FILE, VM_ANON, VM_MMAP };
\end{verbatim}
Defines the type of virtual memory page (binary, file-backed, anonymous, or memory-mapped).

\begin{verbatim}
struct vm_entry {
  void *vaddr;
  bool writable;
  enum vm_type type;
  struct file *file;
  off_t offset;
  size_t read_bytes;
  size_t zero_bytes;
  bool loaded;
  struct hash_elem helem;
};
\end{verbatim}
Represents a virtual page in the supplemental page table, storing metadata for loading and tracking state.

\begin{verbatim}
struct supplemental_page_table {
  struct hash pages;
  struct lock page_lock;
};
\end{verbatim}
Per-process hash table for tracking all virtual pages and synchronizing access.

\textbf{From thread.h:}
\begin{verbatim}
struct supplemental_page_table spt;
\end{verbatim}
Added to each thread to provide a per-process supplemental page table.
\end{quote}


\subsection{Algorithms}

\textbf{Q4:} In a few paragraphs, describe your code for accessing the data
stored in the SPT about a given page.

\begin{quote}
In our implementation, the Supplemental Page Table (SPT) is a hash table used to keep track of virtual pages. It maps virtual addresses to their respective metadata using a hash function on the address.

The VM system supports several page types: binary executables (VM\_BIN), anonymous pages (VM\_ANON), and memory-mapped files (VM\_FILE, VM\_MMAP). Each has distinct handling behavior when loaded.

When a page fault occurs, the handler checks whether the address is valid and looks it up in the SPT. Depending on its type, it either grows the stack, loads it from a file, or restores it from swap.

The frame table holds metadata for physical frames, tracks usage, and helps select a victim frame using the clock algorithm when eviction is necessary.

Pages swapped out are tracked by the swap system, which maintains a bitmap of swap slots. The swap partition is accessed through the block device interface.

For memory-mapped files, mmap and munmap are used. These functions create or remove mappings and ensure file changes are written back when needed.

Stack growth is handled by detecting faults near the stack pointer and allocating new pages accordingly.
\end{quote}


\textbf{Q5:} How does your code coordinate accessed and dirty bits between kernel and user virtual addresses that alias a single frame, or alternatively how do you avoid the issue?

\begin{quote}
In our implementation, the coordination of accessed and dirty bits between kernel and user aliases of a frame is handled by a centralized structure.

Each frame is tracked using a frame table, which includes a bitmap and reverse mapping (frame\_bitmap, frame\_kpages, and frame\_set\_rev\_map). This structure ensures a single point of truth for each physical frame regardless of how it's accessed.

By maintaining reverse mappings from frames to virtual memory entries, we can properly locate the page table entry and synchronize the accessed and dirty bits, even if the modification occurs through a kernel mapping.

This design avoids inconsistencies and prevents the need for separate synchronization between kernel and user views of the same memory.
\end{quote}

\subsection{Synchronization}

\textbf{Q6:} When two user processes both need a new frame at the same time, how are races avoided?

\begin{quote}
Our implementation avoids race conditions in frame allocation by using a global lock. Before a process accesses the frame bitmap or table, it acquires this lock. This ensures that only one process can allocate or evict frames at a time.

Once the frame is allocated, the lock is released, allowing other processes to continue. This approach guarantees mutual exclusion in critical sections, preventing inconsistencies in the frame table.
\end{quote}

\subsection{Rationale}
\textbf{Q7:} Why did you choose the data structure(s) that you did for representing virtual-to-physical mappings?

\begin{quote}
We chose a hash table as the primary data structure for our supplemental page table (SPT) because it provides \texttt{O(1)} average-case lookup time, which is critical for handling page faults efficiently. Since page faults occur frequently in a virtual memory system, fast lookups directly impact overall system performance.

For the frame table, we implemented a bitmap-based structure combined with an array of kernel page pointers (\texttt{frame\_kpages}). The bitmap efficiently tracks which frames are free or allocated with minimal memory overhead, while the array provides direct access to frame data. This combination allows for \texttt{O(1)} frame allocation and deallocation operations.

Our swap system uses a bitmap to track free swap slots, providing constant-time allocation and deallocation of swap space. This is important for quick eviction decisions when memory pressure is high.

We considered alternatives like balanced trees (\texttt{O(log n)} lookup) for the SPT or linked lists for frame tracking, but we discarded the tree option since Pintos does not include a balanced tree implementation by default, and linked lists were not suitable due to their high memory overhead. The constant-time operations of hash tables and bitmaps made them optimal choices for our virtual memory implementation where performance is critical, especially during page faults and evictions.
\end{quote}


\section{Paging to and from Disk}

\textbf{Q8:} Copy here the declaration of each new or changed `struct' or `struct' member, global or static variable, `typedef', or enumeration.  Identify the purpose of each in 25 words or less.
\begin{quote}
\textbf{From page.h:}
\begin{verbatim}
struct vm_entry {
  int swap_slot;
};
\end{verbatim}
Holds the swap slot index used when a page is evicted to swap.

\textbf{From swap.c:}
\begin{verbatim}
static struct block *swap_block;
static struct bitmap *swap_bitmap;
static struct lock swap_lock;
static size_t swap_slots;
\end{verbatim}
Swap system globals: swap block device, bitmap to track slot usage, lock for synchronization, and total swap slot count.
\end{quote}


\pagebreak

\subsection{Algorithms}

\textbf{Q9:} When a frame is required but none is free, some frame must be evicted.  Describe your code for choosing a frame to evict.
\begin{quote}
Our implementation uses the clock algorithm for frame eviction. When no free frames are available, we:

\begin{itemize}
  \item Maintain a global clock hand that points to the next frame to consider for eviction
  \item Examine the frame at the current clock position
  \item Using \texttt{pagedir\_is\_accessed()}, we check if the page has been accessed recently by examining the accessed bit in the page table entry
  \item If the accessed bit is set (page was recently used):
  \begin{itemize}
    \item Call \texttt{pagedir\_set\_accessed()} to clear the accessed bit
    \item Advance the clock hand to the next frame
    \item Continue the search
  \end{itemize}
  \item If the accessed bit is clear (page wasn't recently used):
  \begin{itemize}
    \item This frame is selected for eviction
    \item If the page is dirty, call \texttt{swap\_out()}
    \item Return the frame for reuse
  \end{itemize}
\end{itemize}

This approach approximates a Least Recently Used (LRU) policy while being much more efficient to implement, as it only requires a single pass through the frames to find an eviction candidate.
\end{quote}




\textbf{Q10:} When a process P obtains a frame that was previously used by a process Q, how do you adjust the page table (and any other data structures) to reflect the frame Q no longer has?
\begin{quote}
When a process \texttt{P} obtains a frame that was previously used by another process \texttt{Q}, the code first identifies the frame’s current owner using the \texttt{frame\_rev\_map}. If the frame belongs to \texttt{Q} and the page is dirty, it writes the content to swap space using \texttt{swap\_out} and updates the corresponding \texttt{vm\_entry} with the swap slot. Then, it clears \texttt{Q}'s page table entry using \texttt{pagedir\_clear\_page}, effectively removing \texttt{Q}'s access to that frame. After that, it updates the frame table by setting \texttt{frame\_rev\_map[idx]} and \texttt{frame\_kpages[idx]} to \texttt{NULL}, indicating that the frame is no longer associated with \texttt{Q}. This process ensures that \texttt{Q} no longer owns or accesses the frame, allowing \texttt{P} to safely reuse it. Later, when \texttt{P} installs its own page into this frame, the frame table is updated again to reflect \texttt{P}’s ownership.
\end{quote}


\textbf{Q11:} Explain your heuristic for deciding whether a page fault for an invalid virtual address should cause the stack to be extended into the page that faulted.
\begin{quote}
In our implementation, we decide to grow the stack only if two conditions are met. First, the faulting address must be in the user address space, which means it must be below \texttt{PHYS\_BASE}. Second, the faulting address must be close to the current stack pointer—specifically, no more than 32 bytes below it. This 32-byte margin is used to allow for common stack operations like pushing arguments or return addresses. If both conditions are satisfied, we treat the fault as a valid stack growth request and allocate a new page. Otherwise, we consider it an invalid access and terminate the process.
\end{quote}


\subsection{Synchronization}

\textbf{Q12:} Explain the basics of your VM synchronization design.  In particular, explain how it prevents deadlock.  (Refer to the textbook for an explanation of the necessary conditions for deadlock.)
\begin{quote}
In our VM implementation, we use separate locks for each shared resource to ensure safe access without deadlocks. The global frame table is protected by \texttt{vm\_lock}, each thread’s supplemental page table uses its own \texttt{page\_lock}, the swap system uses \texttt{swap\_lock}, and file operations are guarded by \texttt{file\_lock}. Our code holds locks for the minimum time necessary, releasing them as soon as the critical section is complete. This reduces the chance of resource contention. And for operations that need to be atomic (like frame allocation and page table updates), we use a single lock to protect the entire operation.
\end{quote}


\textbf{Q13:} A page fault in process P can cause another process Q's frame to be evicted. How do you ensure that Q cannot access or modify the page during the eviction process?  How do you avoid a race between P evicting Q's frame and Q faulting the page back in?
\begin{quote}
To prevent races, P evicts Q’s frame while holding \texttt{vm\_lock}. It clears Q’s page table entry, so Q can’t access the frame anymore. Since both P and Q must hold \texttt{vm\_lock} to modify page mappings, Q can’t remap or access the frame until P finishes. This ensures safe eviction without conflicts.
\end{quote}

\newpage
\textbf{Q14:} Suppose a page fault in process P causes a page to be read from the file system or swap.  How do you ensure that a second process Q cannot interfere by e.g. attempting to evict the frame while it is still being read in?
\begin{quote}
To prevent Q from evicting a frame while P is reading a page into it, we use locks. P first reserves the frame under \texttt{vm\_lock}, so no other process can evict it. Then, P reads the data from swap or file while holding the appropriate lock (\texttt{swap\_lock} or \texttt{file\_lock}). During this time, the frame isn’t visible in \texttt{frame\_rev\_map}, so Q won’t choose it for eviction. After the read is complete, P updates \texttt{frame\_rev\_map} and the page table under \texttt{vm\_lock}. This locking order ensures Q cannot interfere during the loading process.
\end{quote}


\textbf{Q15:} Explain how you handle access to paged-out pages that occur during system calls.  Do you use page faults to bring in pages (as in user programs), or do you have a mechanism for "locking" frames into physical memory, or do you use some other design?  How do you gracefully handle attempted accesses to invalid virtual addresses?
\begin{quote}
In our design, system calls use the same page fault mechanism as user programs. If a syscall accesses a paged-out user address, it triggers a page fault. The fault handler loads the page from swap or file, updates the page table, and resumes the syscall. We don’t lock pages during syscalls—frames can still be evicted safely using \texttt{vm\_lock}. If the address is invalid and not in the supplemental page table (or not eligible for stack growth), we terminate the process with \texttt{kill()}.
\end{quote}


\subsection{Rationale}

\textbf{Q16:} A single lock for the whole VM system would make synchronization easy, but limit parallelism.  On the other hand, using many locks complicates synchronization and raises the possibility for deadlock but allows for high parallelism.  Explain where your design falls along this continuum and why you chose to design it this way.
\begin{quote}
Our design uses a few medium-sized locks instead of one big lock or many small ones. We have separate locks for the frame table (\texttt{vm\_lock}), each process’s supplemental page table (\texttt{page\_lock}), the swap system (\texttt{swap\_lock}), and file I/O (\texttt{file\_lock}). This way, different parts of the VM system can work in parallel without blocking each other too much. For example, two processes can handle their own page faults at the same time, and file reads or swap operations don’t block frame allocation. We chose this design to keep things simple and safe while still allowing some parallelism. Using one big lock would be easier but too slow, and using many small locks would be too complex and risky for deadlocks. Our approach gives a good balance between performance and safety.
\end{quote}



\section{Memory Mapped Files}

\subsection{Data Structures}

\textbf{Q17:} Copy here the declaration of each new or changed `struct' or `struct' member, global or static variable, `typedef', or enumeration.  Identify the purpose of each in 25 words or less.
\begin{quote}
\textbf{From page.h:}
\begin{verbatim}
enum vm_type { VM_BIN, VM_FILE, VM_ANON, VM_MMAP };
\end{verbatim}
Page type for memory-mapped file pages.

\begin{verbatim}
struct vm_entry {
  int mapid;
};
\end{verbatim}
Associates this page with a specific memory-mapped file region via map ID.

\begin{verbatim}
struct mmap_desc {
  int mapid;
  struct file *file;
  void *base_addr;
  size_t page_cnt;
  struct list_elem elem;
};
\end{verbatim}
Describes a mapped file region, including backing file, base virtual address, and number of pages.
\end{quote}


\subsection{Algorithms}

\textbf{Q18:} Describe how memory mapped files integrate into your virtual memory subsystem.  Explain how the page fault and eviction processes differ between swap pages and other pages.
\begin{quote}
Memory-mapped files are handled as \texttt{VM\_MMAP} pages in our virtual memory system. When a process maps a file with \texttt{sys\_mmap()}, we create \texttt{vm\_entry} structures for each page and mark them as not loaded. These entries are added to the supplemental page table.

When the process accesses a mapped page for the first time, a page fault occurs. If the page was evicted earlier and saved in swap, we load it back using \texttt{swap\_in()}. If not, we read the data from the file using \texttt{file\_read\_at()} and zero the rest of the page. After loading, we install the page and mark it as loaded.

If a memory-mapped page is evicted, we write it to swap just like a regular anonymous page. But when the process calls \texttt{sys\_munmap()}, we check if the page was modified. If it was, we write the changes back to the file using \texttt{file\_write\_at()}. Then we clear the page, free the frame, and release the swap slot if used.
\end{quote}


\textbf{Q19:} Explain how you determine whether a new file mapping overlaps any existing segment.
\begin{quote}
To check for overlaps when mapping a new file, we first calculate how many pages the file will use. Then, we make sure the starting address is page-aligned and the entire range stays within the user address space. After that, we check each page in the range one by one using a function like \texttt{page\_already\_mapped()}, which looks it up in the supplemental page table. If any of the pages are already mapped, it means there’s an overlap with an existing segment, so we return an error and cancel the mapping.
\end{quote}


\subsection{Rationale}
\textbf{Q20:} Mappings created with "mmap" have similar semantics to those of data demand-paged from executables, except that "mmap" mappings are written back to their original files, not to swap.  This implies that much of their implementation can be shared.  Explain why your implementation either does or does not share much of the code for the two situations.

\begin{quote}
In our code, memory-mapped files (\texttt{VM\_MMAP}) and executable-backed pages (\texttt{VM\_BIN} or \texttt{VM\_FILE}) use almost the same logic. When a page fault happens, we handle both types the same way: we allocate a frame, read the needed bytes from the file using \texttt{file\_read\_at()}, zero the rest if needed, and install the page with \texttt{install\_page()}. Eviction also works the same—if the page is dirty, we call \texttt{swap\_out()} and save it to swap. The only real difference is during \texttt{munmap()}: for \texttt{VM\_MMAP} pages, we check if the page was changed, and if so, we write it back to the file using \texttt{file\_write\_at()}. So overall, most of the implementation is shared, and only the write-back step for mmap pages is handled separately.
\end{quote}




\section{Survey Questions}

Answering these questions is optional, but it will help us improve the course in future quarters.  Feel free to tell us anything you want--these questions are just to spur your thoughts.  You may also choose to respond anonymously in the course evaluations at the end of the quarter.

\textbf{Q1:} In your opinion, was this assignment, or any one of the three problems in it, too easy or too hard?  Did it take too long or too little time?
\begin{quote}
  Answer here
\end{quote}

\textbf{Q2:} Did you find that working on a particular part of the assignment gave you greater insight into some aspect of OS design?
\begin{quote}
  Answer here
\end{quote}

\textbf{Q3:} Is there some particular fact or hint we should give students in future quarters to help them solve the problems?  Conversely, did you find any of our guidance to be misleading?
\begin{quote}
  Answer here
\end{quote}

\textbf{Q4:} Do you have any suggestions for the TAs to more effectively assist students, either for future quarters or the remaining projects?
\begin{quote}
  Answer here
\end{quote}

\textbf{Q5:} Any other comments?
\begin{quote}
  Answer here
\end{quote}


\end{document}